\documentclass[11pt,a4paper]{article}

% Packages
\usepackage[utf8]{inputenc}
\usepackage[T1]{fontenc}
\usepackage{amsmath,amssymb,amsfonts}
\usepackage{graphicx}
\usepackage{booktabs}
\usepackage{hyperref}
\usepackage[margin=2.5cm]{geometry}
\usepackage{float}
\usepackage{subcaption}
\usepackage{xcolor}
\usepackage{listings}
\usepackage{fancyhdr}
\usepackage{titlesec}

% Page style
\pagestyle{fancy}
\fancyhf{}
\rhead{VELO Toy Model Study}
\lhead{Track Reconstruction Analysis}
\rfoot{Page \thepage}

% Title formatting
\titleformat{\section}{\Large\bfseries}{\thesection}{1em}{}
\titleformat{\subsection}{\large\bfseries}{\thesubsection}{1em}{}

% Custom commands
\newcommand{\sres}{\sigma_{\text{res}}}
\newcommand{\sscatt}{\sigma_{\text{scatt}}}
\newcommand{\eps}{\varepsilon}

% Document info
\title{%
    \vspace{-2cm}
    \textbf{VELO Toy Model: Track Reconstruction Study} \\[0.5em]
    \large Sparse and Dense Event Analysis \\[0.3em]
    \normalsize Parameter Optimization and Performance Characterization
}
\author{George William Scriven}
\date{December 2025}

\begin{document}

\maketitle

\begin{abstract}
This report documents a comprehensive study of track reconstruction performance using the VELO 
toy model with a Hamiltonian-based segment matching algorithm. We investigate the effects of 
hit resolution ($\sres$), multiple scattering ($\sscatt$), scale factor ($n$), and threshold 
function (step vs.\ erf) on reconstruction efficiency and ghost rate. Three experimental runs 
(Runs 8, 9, and 10) were conducted, analyzing over 3000 events across sparse (10 tracks) and 
dense (100 tracks) configurations. Key findings include: (1) optimal performance at 
$\sres \leq 10~\mu$m and $\sscatt \leq 0.2$~mrad, (2) scale factor $n=3$--5 provides best 
efficiency-ghost trade-off, and (3) performance degradation with increasing multiple scattering 
is driven by increased track angular density. The validator uses a completeness threshold of 
70\% as specified in the LHCb instructions.
\end{abstract}

\tableofcontents
\newpage

%=============================================================================
\section{Introduction}
%=============================================================================

\subsection{Motivation}

The LHCb VELO (VErtex LOcator) detector requires efficient track reconstruction algorithms 
capable of handling high track multiplicities. This study uses a toy model to investigate 
the fundamental limits and parameter dependencies of a Hamiltonian-based track reconstruction 
approach.

\subsection{The VELO Toy Model}

The toy model simulates:
\begin{itemize}
    \item A simplified detector geometry with multiple parallel sensor planes
    \item Particle tracks traversing the detector with configurable multiplicity
    \item Hit position smearing based on measurement resolution ($\sres$)
    \item Multiple scattering effects parameterized by $\sscatt$
\end{itemize}

\subsection{Hamiltonian Track Reconstruction}

Track reconstruction is formulated as an optimization problem using a Hamiltonian approach:
\begin{equation}
    H = -\sum_{\text{segments}} w_{ij} \cdot x_i \cdot x_j
\end{equation}
where $w_{ij}$ represents the compatibility weight between segment pairs $(i,j)$ and $x_i \in \{0,1\}$ 
indicates segment selection.

\subsubsection{Angular Threshold}

The segment compatibility is determined by the inter-segment angle $\theta$ compared to a 
threshold $\eps$:
\begin{equation}
    \eps = \sqrt{\theta_s^2 + \theta_r^2 + \theta_{\min}^2}
    \label{eq:epsilon}
\end{equation}
where:
\begin{itemize}
    \item $\theta_s = n \cdot \sscatt$ is the multiple scattering contribution
    \item $\theta_r = \arctan(n \cdot \sres / \Delta z)$ is the resolution contribution
    \item $\theta_{\min} \approx 15~\mu$rad is a minimum threshold
    \item $n$ is the scale factor
\end{itemize}

\subsubsection{Threshold Functions}

Two threshold functions are compared:

\textbf{Step function:}
\begin{equation}
    w(\theta) = \begin{cases}
        1 & \text{if } |\theta| \leq \eps \\
        0 & \text{otherwise}
    \end{cases}
\end{equation}

\textbf{Error function (erf):}
\begin{equation}
    w(\theta) = \frac{1}{2}\left(1 + \text{erf}\left(\frac{\eps - |\theta|}{\theta_d \sqrt{2}}\right)\right)
\end{equation}
where $\theta_d$ is the erf smoothing parameter (erf\_sigma).

%=============================================================================
\section{Experimental Setup}
%=============================================================================

\subsection{Parameter Space}

Three comprehensive parameter scans were conducted:

\begin{table}[H]
\centering
\caption{Parameter ranges for each experimental run}
\label{tab:params}
\begin{tabular}{lccc}
\toprule
\textbf{Parameter} & \textbf{Runs 8} & \textbf{Runs 9} & \textbf{Runs 10} \\
\midrule
$\sres$ ($\mu$m) & 1, 10 & 1--200 & 0, 10, 20, 50 \\
$\sscatt$ (mrad) & 0.01 & 0.001--1 & 0, 0.1, 0.2, 0.3, 0.5, 1.0 \\
Scale $n$ & 2--10 & 2--10 & 3, 4, 5 \\
erf\_sigma & $10^{-6}$--$10^{-3}$ & $10^{-6}$--$10^{-3}$ & $10^{-4}$ \\
Track density & sparse (10), dense (100) & sparse (10), dense (100) & sparse (10) \\
Threshold & step, erf & step, erf & step \\
Events per config & 10 & 10 & 50--500 \\
\bottomrule
\end{tabular}
\end{table}

\subsection{Validation Metrics}

Track reconstruction quality is evaluated using metrics defined in the LHCb instructions:

\textbf{Completeness:} Fraction of truth track hits found in reconstructed track
\begin{equation}
    \text{Completeness} = \frac{N_{\text{shared}}}{N_{\text{truth}}}
\end{equation}

\textbf{Track Classification:}
\begin{itemize}
    \item \textbf{GOOD}: Reconstructed track with completeness $\geq 70\%$
    \item \textbf{GHOST}: Reconstructed track with completeness $< 70\%$
\end{itemize}

\textbf{Efficiency:}
\begin{equation}
    \text{Efficiency} = \frac{N_{\text{GOOD}}}{N_{\text{truth}}}
\end{equation}

\textbf{Ghost Rate:}
\begin{equation}
    \text{Ghost Rate} = \frac{N_{\text{GHOST}}}{N_{\text{reconstructed}}}
\end{equation}

\subsection{Computing Infrastructure}

All simulations were executed on the Nikhef HTCondor cluster with:
\begin{itemize}
    \item Batch job submission for parallel parameter scanning
    \item Event stores saved as compressed pickle files
    \item Post-processing aggregation of metrics
\end{itemize}

%=============================================================================
\section{Runs 8: Scale Factor and ERF Sigma Study}
%=============================================================================

\subsection{Objectives}

Runs 8 investigated:
\begin{enumerate}
    \item The effect of scale factor $n$ on reconstruction performance
    \item Comparison of step vs.\ erf threshold functions
    \item Independence of step function results from erf\_sigma parameter
    \item Performance difference between sparse (10 tracks) and dense (100 tracks) events
\end{enumerate}

\subsection{Key Results}

\begin{figure}[H]
\centering
\includegraphics[width=0.95\textwidth]{runs_8/scale_effect.png}
\caption{Runs 8: Effect of scale factor on track efficiency and ghost rate for sparse and dense events}
\label{fig:runs8_scale}
\end{figure}

\subsubsection{Step Function Independence Verification}

A critical verification was performed to confirm that step function results are 
independent of the erf\_sigma parameter:

\begin{quote}
\textit{Result: Step function gives IDENTICAL results for all erf\_sigma values, 
with maximum efficiency variation $< 10^{-10}$ across the same (parameters, repeat) 
combinations.}
\end{quote}

This confirms correct implementation of the threshold logic.

\begin{figure}[H]
\centering
\includegraphics[width=0.8\textwidth]{runs_8/erf_sigma_effect.png}
\caption{Runs 8: ERF sigma effect on track efficiency (confirms step function independence)}
\label{fig:runs8_erf}
\end{figure}

\subsubsection{Sparse Events Performance}

\begin{table}[H]
\centering
\caption{Best configurations for sparse events (Runs 8)}
\label{tab:runs8_sparse}
\begin{tabular}{lcc}
\toprule
\textbf{Metric} & \textbf{Step Function} & \textbf{ERF Function} \\
\midrule
Best efficiency & 93.8\% $\pm$ 1.2\% & 94.2\% $\pm$ 1.5\% \\
Optimal scale & 5 & 5 \\
Ghost rate & 0.8\% & 0.6\% \\
\bottomrule
\end{tabular}
\end{table}

\subsubsection{Dense Events Performance}

\begin{table}[H]
\centering
\caption{Best configurations for dense events (Runs 8)}
\label{tab:runs8_dense}
\begin{tabular}{lcc}
\toprule
\textbf{Metric} & \textbf{Step Function} & \textbf{ERF Function} \\
\midrule
Best efficiency & 13.8\% $\pm$ 2.4\% & 12.5\% $\pm$ 3.1\% \\
Optimal scale & 2 & 3 \\
Ghost rate & 8.2\% & 9.4\% \\
\bottomrule
\end{tabular}
\end{table}

\subsubsection{Key Finding: Density Limitation}

Dense events (100 tracks) show significantly reduced performance compared to sparse events 
(10 tracks). This is a fundamental limitation due to:
\begin{itemize}
    \item Increased track merging (multiple truth tracks combined into one reconstructed track)
    \item Higher combinatorial background from false segment pairs
    \item Violation of the ``one track per segment'' assumption in high-density regions
\end{itemize}

%=============================================================================
\section{Runs 9: Extended Parameter Scan}
%=============================================================================

\subsection{Objectives}

Runs 9 extended the parameter space to explore:
\begin{enumerate}
    \item Hit resolution from 1--200 $\mu$m
    \item Multiple scattering from 0.001--1 mrad
    \item Full scale factor range 2--10
    \item Both sparse and dense configurations
\end{enumerate}

\subsection{Hit Resolution Scan}

\begin{figure}[H]
\centering
\includegraphics[width=0.95\textwidth]{runs_9/hit_resolution_scan.png}
\caption{Effect of hit resolution on reconstruction performance for step and ERF methods}
\label{fig:res_scan}
\end{figure}

Key observations:
\begin{itemize}
    \item Efficiency decreases monotonically with increasing $\sres$
    \item Ghost rate increases with $\sres$ due to wider acceptance windows
    \item Performance degrades rapidly above $\sres = 50~\mu$m
    \item Sparse events maintain $>50\%$ efficiency up to $\sres = 100~\mu$m
\end{itemize}

\subsection{Multiple Scattering Scan}

\begin{figure}[H]
\centering
\includegraphics[width=0.95\textwidth]{runs_9/scattering_scan.png}
\caption{Effect of multiple scattering on reconstruction performance}
\label{fig:scatter_scan}
\end{figure}

Key observations:
\begin{itemize}
    \item Efficiency drops significantly for $\sscatt > 0.2$~mrad
    \item Ghost rate increases from $\sim 5\%$ to $>30\%$ as $\sscatt$ increases
    \item Dense events are more sensitive to scattering than sparse events
    \item The erf function shows slightly better performance at high scattering
\end{itemize}

\subsection{Scale Factor Optimization}

\begin{figure}[H]
\centering
\includegraphics[width=0.95\textwidth]{runs_9/scale_scan.png}
\caption{Scale factor optimization: efficiency and ghost rate vs.\ scale factor}
\label{fig:scale_scan}
\end{figure}

\begin{table}[H]
\centering
\caption{Optimal scale factors by configuration (Runs 9)}
\label{tab:optimal_scale}
\begin{tabular}{lcc}
\toprule
\textbf{Configuration} & \textbf{Optimal Scale} & \textbf{Peak Efficiency} \\
\midrule
Sparse + Step & 5 & 94.1\% \\
Sparse + ERF & 5 & 93.8\% \\
Dense + Step & 3 & 14.2\% \\
Dense + ERF & 3 & 13.1\% \\
\bottomrule
\end{tabular}
\end{table}

%=============================================================================
\section{Runs 10: Instruction-Aligned Parameters}
%=============================================================================

\subsection{Objectives}

Runs 10 used the exact parameter values specified in Instructions.pdf:
\begin{itemize}
    \item $\sres \in \{0, 10, 20, 50\}~\mu$m (default: 10 $\mu$m)
    \item $\sscatt \in \{0, 0.1, 0.2, 0.3, 0.5, 1.0\}$~mrad (default: 0.1 mrad)
    \item Scale $n \in \{3, 4, 5\}$
    \item High statistics: 50--500 events per configuration
\end{itemize}

\begin{figure}[H]
\centering
\includegraphics[width=0.95\textwidth]{runs_10/runs10_analysis.png}
\caption{Runs 10: Comprehensive analysis of instruction-aligned parameter scan}
\label{fig:runs10_analysis}
\end{figure}

\subsection{Results Summary}

\begin{table}[H]
\centering
\caption{Complete results table (Runs 10, sparse events)}
\label{tab:runs10_results}
\begin{tabular}{rrrrrr}
\toprule
$\sres$ ($\mu$m) & $\sscatt$ (mrad) & Scale & Efficiency (\%) & Ghost (\%) & $N$ \\
\midrule
0 & 0.10 & 5 & 92.0 & 3.9 & 50 \\
10 & 0.00 & 5 & 66.8 & 16.7 & 50 \\
10 & 0.10 & 3 & 81.4 & 7.7 & 50 \\
10 & 0.10 & 4 & 76.6 & 11.5 & 50 \\
10 & 0.10 & 5 & 64.5 & 18.2 & 500 \\
10 & 0.20 & 5 & 64.2 & 16.9 & 50 \\
10 & 0.30 & 5 & 60.2 & 20.7 & 50 \\
10 & 0.50 & 5 & 45.6 & 28.4 & 50 \\
10 & 1.00 & 5 & 27.4 & 40.4 & 50 \\
20 & 0.10 & 5 & 40.0 & 31.4 & 50 \\
50 & 0.10 & 5 & 11.4 & 65.7 & 50 \\
\bottomrule
\end{tabular}
\end{table}

\subsection{Performance at Instruction Defaults}

At the instruction-specified default parameters ($\sres = 10~\mu$m, $\sscatt = 0.1$~mrad):

\begin{table}[H]
\centering
\caption{Performance at instruction defaults}
\label{tab:defaults}
\begin{tabular}{ccc}
\toprule
\textbf{Scale} & \textbf{Efficiency} & \textbf{Ghost Rate} \\
\midrule
3 & 81.4\% & 7.7\% \\
4 & 76.6\% & 11.5\% \\
5 & 64.5\% & 18.2\% \\
\bottomrule
\end{tabular}
\end{table}

\textbf{Recommendation:} Scale = 3 provides the best efficiency-ghost trade-off at 
instruction defaults.

\begin{figure}[H]
\centering
\includegraphics[width=0.95\textwidth]{runs_10/runs10_heatmaps.png}
\caption{Runs 10: Efficiency heatmaps showing performance across resolution and scattering parameter space for each scale factor}
\label{fig:runs10_heatmaps}
\end{figure}

\begin{figure}[H]
\centering
\includegraphics[width=0.95\textwidth]{runs_10/example_tracks.png}
\caption{Example track visualizations from Runs 10: high efficiency (top) vs.\ low efficiency (bottom) events}
\label{fig:runs10_tracks}
\end{figure}

%=============================================================================
\section{Track Angular Density Analysis}
%=============================================================================

\subsection{The Degradation Mechanism}

The degradation in track reconstruction efficiency with increasing multiple scattering 
is fundamentally due to \textbf{increased effective track density in angular space}.

\subsubsection{Angular Threshold Growth}

From Equation~\ref{eq:epsilon}, as $\sscatt$ increases:
\begin{equation}
    \eps(\sscatt) = \sqrt{(n \cdot \sscatt)^2 + \theta_r^2 + \theta_{\min}^2}
\end{equation}

At $\sres = 10~\mu$m and $n = 5$:

\begin{table}[H]
\centering
\caption{Angular threshold vs.\ multiple scattering}
\label{tab:threshold}
\begin{tabular}{cc}
\toprule
$\sscatt$ (mrad) & $\eps$ (mrad) \\
\midrule
0.0 & 1.52 \\
0.1 & 1.60 \\
0.2 & 1.83 \\
0.5 & 2.89 \\
1.0 & 5.15 \\
\bottomrule
\end{tabular}
\end{table}

\subsubsection{False Acceptance Problem}

The wider threshold $\eps$ required to capture scattered tracks also accepts more 
false segment combinations:

\begin{equation}
    P(\text{false accepted}) \propto \frac{2\eps}{\theta_{\max}}
\end{equation}

where $\theta_{\max}$ is the angular range of false combinations.

\subsection{Segment Angle Distributions}

Analysis of inter-segment angles from actual events reveals:

\begin{itemize}
    \item \textbf{True pairs:} Angles follow a narrow distribution centered near zero, 
          with width increasing with $\sscatt$
    \item \textbf{False pairs:} Angles are broadly distributed across the full angular range
    \item \textbf{Overlap:} As $\sscatt$ increases, the true distribution broadens and 
          overlaps more with the false distribution
\end{itemize}

\begin{figure}[H]
\centering
\includegraphics[width=0.95\textwidth]{segment_angle_histograms.png}
\caption{Segment angle distributions showing true (blue) vs.\ false (red) pair separation at different $\sscatt$ values, with step and erf threshold overlays}
\label{fig:angle_hist}
\end{figure}

\begin{figure}[H]
\centering
\includegraphics[width=0.95\textwidth]{track_density_analysis.png}
\caption{Track angular density analysis: threshold functions, false acceptance rate, and ghost rate vs.\ multiple scattering}
\label{fig:density_analysis}
\end{figure}

\begin{figure}[H]
\centering
\includegraphics[width=0.95\textwidth]{angle_threshold_summary.png}
\caption{Summary of angular thresholds: comparison of true track angle distributions at low vs.\ high scattering with step and erf acceptance curves}
\label{fig:angle_summary}
\end{figure}

\subsection{Summary Statistics}

\begin{table}[H]
\centering
\caption{Angular statistics vs.\ multiple scattering (from event data)}
\label{tab:angle_stats}
\begin{tabular}{cccc}
\toprule
$\sscatt$ (mrad) & Mean $|\theta|$ (mrad) & Std $|\theta|$ (mrad) & Threshold $\eps$ (mrad) \\
\midrule
0.0 & 0.12 & 0.08 & 1.52 \\
0.1 & 0.15 & 0.12 & 1.60 \\
0.2 & 0.21 & 0.18 & 1.83 \\
0.5 & 0.38 & 0.31 & 2.89 \\
1.0 & 0.72 & 0.55 & 5.15 \\
\bottomrule
\end{tabular}
\end{table}

%=============================================================================
\section{Ghost Classification Analysis}
%=============================================================================

\subsection{Types of Ghost Tracks}

The validator classifies reconstructed tracks as GHOST when completeness $< 70\%$. 
Analysis reveals two categories:

\begin{enumerate}
    \item \textbf{Partial Real Ghosts:} Incomplete reconstructions of actual tracks
          \begin{itemize}
              \item High purity ($>50\%$) but low completeness
              \item Caused by track splitting due to scattering
              \item More common at high $\sscatt$
          \end{itemize}
    
    \item \textbf{Pure Fake Ghosts:} False combinations with no strong truth match
          \begin{itemize}
              \item Low purity, no dominant truth track
              \item Caused by combinatorial background
              \item More common at high resolution uncertainty
          \end{itemize}
\end{enumerate}

\subsection{Ghost Composition Analysis}

\begin{table}[H]
\centering
\caption{Ghost composition at different scattering values}
\label{tab:ghost_composition}
\begin{tabular}{ccccc}
\toprule
$\sscatt$ (mrad) & $N_{\text{ghost}}$ & Partial Real & Pure Fake & Avg.\ Purity \\
\midrule
0.1 & 12 & 8 (67\%) & 4 (33\%) & 62\% \\
0.5 & 28 & 18 (64\%) & 10 (36\%) & 48\% \\
1.0 & 45 & 24 (53\%) & 21 (47\%) & 41\% \\
\bottomrule
\end{tabular}
\end{table}

\textbf{Key insight:} At all scattering values, a significant fraction of ``ghosts'' 
are actually partial reconstructions of real tracks. This is correct behavior per 
LHCb standards---partial tracks should not be counted as successful reconstructions.

%=============================================================================
\section{Step vs.\ ERF Threshold Comparison}
%=============================================================================

\subsection{Mathematical Comparison}

The step and erf thresholds differ in their treatment of segments near the boundary:

\begin{itemize}
    \item \textbf{Step:} Hard cutoff at $|\theta| = \eps$
    \item \textbf{ERF:} Soft transition with width controlled by $\theta_d$
\end{itemize}

\subsection{Performance Comparison}

\begin{table}[H]
\centering
\caption{Step vs.\ ERF performance comparison (average across all configurations)}
\label{tab:step_vs_erf}
\begin{tabular}{lccc}
\toprule
\textbf{Configuration} & \textbf{Step Eff.} & \textbf{ERF Eff.} & \textbf{Difference} \\
\midrule
Sparse, low $\sscatt$ & 87.2\% & 86.8\% & +0.4\% \\
Sparse, high $\sscatt$ & 45.1\% & 46.3\% & $-$1.2\% \\
Dense, low $\sscatt$ & 12.4\% & 11.8\% & +0.6\% \\
Dense, high $\sscatt$ & 5.2\% & 5.8\% & $-$0.6\% \\
\bottomrule
\end{tabular}
\end{table}

\textbf{Conclusion:} The step function performs comparably to the erf function across 
all tested configurations. The step function is recommended for simplicity and 
computational efficiency.

\begin{figure}[H]
\centering
\includegraphics[width=0.8\textwidth]{runs_9/step_vs_erf.png}
\caption{Direct comparison of step vs.\ ERF threshold function performance}
\label{fig:step_vs_erf}
\end{figure}

%=============================================================================
\section{Conclusions and Recommendations}
%=============================================================================

\subsection{Key Findings}

\begin{enumerate}
    \item \textbf{Sparse events achieve high efficiency:} Up to 94\% efficiency with 
          $<1\%$ ghost rate at optimal parameters
    
    \item \textbf{Dense events are fundamentally limited:} Maximum $\sim14\%$ efficiency 
          due to track merging and combinatorial background
    
    \item \textbf{Multiple scattering drives degradation:} Performance degrades due to 
          increased track angular density, not algorithmic limitations
    
    \item \textbf{Step function is sufficient:} No significant benefit from erf smoothing
    
    \item \textbf{Scale factor 3--5 is optimal:} Balances track capture vs.\ false acceptance
\end{enumerate}

\subsection{Recommended Parameters}

For LHCb VELO-like conditions:

\begin{table}[H]
\centering
\caption{Recommended parameters for optimal performance}
\label{tab:recommendations}
\begin{tabular}{ll}
\toprule
\textbf{Parameter} & \textbf{Recommended Value} \\
\midrule
Hit resolution $\sres$ & $\leq 10~\mu$m \\
Multiple scattering $\sscatt$ & $\leq 0.2$~mrad \\
Scale factor $n$ & 3--5 (use 3 for lowest ghost rate) \\
Threshold function & Step (for simplicity) \\
Completeness threshold & 70\% (per LHCb standard) \\
\bottomrule
\end{tabular}
\end{table}

\subsection{Performance Summary at Instruction Defaults}

At $\sres = 10~\mu$m, $\sscatt = 0.1$~mrad, scale = 3:
\begin{itemize}
    \item \textbf{Track efficiency:} 81.4\%
    \item \textbf{Ghost rate:} 7.7\%
    \item \textbf{Events analyzed:} 50
\end{itemize}

\begin{figure}[H]
\centering
\includegraphics[width=0.95\textwidth]{metrics_vs_scattering.png}
\caption{Summary: Key metrics (efficiency and ghost rate) vs.\ multiple scattering at best parameters}
\label{fig:metrics_summary}
\end{figure}

\begin{figure}[H]
\centering
\includegraphics[width=0.95\textwidth]{comparison_all_runs.png}
\caption{Performance comparison across all experimental runs (Runs 8, 9, and 10)}
\label{fig:all_runs}
\end{figure}

\begin{figure}[H]
\centering
\includegraphics[width=0.95\textwidth]{example_tracks_comparison.png}
\caption{Visual comparison of track reconstruction: high efficiency events (top row) vs.\ low efficiency events (bottom row)}
\label{fig:tracks_comparison}
\end{figure}

\subsection{Future Work}

\begin{enumerate}
    \item Investigate adaptive thresholds that adjust with local track density
    \item Explore machine learning approaches for segment classification
    \item Study performance with realistic LHCb detector geometry
    \item Implement track fitting to improve ghost rejection
\end{enumerate}

%=============================================================================
\section*{Appendix A: Generated Figures}
%=============================================================================

The following figures were generated during this study and are saved in the 
\texttt{Velo\_toy} directory:

\begin{itemize}
    \item \texttt{runs\_8/erf\_sigma\_effect.png} -- ERF sigma effect heatmap
    \item \texttt{runs\_8/scale\_effect.png} -- Scale factor effect plots
    \item \texttt{runs\_9/hit\_resolution\_scan.png} -- Resolution scan results
    \item \texttt{runs\_9/scattering\_scan.png} -- Scattering scan results
    \item \texttt{runs\_9/scale\_scan.png} -- Scale optimization plots
    \item \texttt{runs\_9/erf\_sigma\_scan.png} -- ERF sigma scan results
    \item \texttt{runs\_10/runs10\_analysis.png} -- Runs 10 summary plots
    \item \texttt{runs\_10/runs10\_heatmaps.png} -- Efficiency heatmaps by scale
    \item \texttt{runs\_10/example\_tracks.png} -- Example track visualizations
    \item \texttt{track\_density\_analysis.png} -- Angular density analysis
    \item \texttt{segment\_angle\_histograms.png} -- Angle distribution histograms
    \item \texttt{angle\_threshold\_summary.png} -- Threshold comparison plots
    \item \texttt{metrics\_vs\_scattering.png} -- Metrics vs.\ scattering summary
    \item \texttt{example\_tracks\_comparison.png} -- High vs.\ low efficiency comparison
    \item \texttt{executive\_summary.png} -- Overall study summary
    \item \texttt{comparison\_all\_runs.png} -- Cross-run comparison
\end{itemize}

%=============================================================================
\section*{Appendix B: Data Files}
%=============================================================================

\begin{itemize}
    \item \texttt{runs\_8/metrics\_fixed.csv} -- Runs 8 aggregated metrics
    \item \texttt{runs\_9/metrics\_fixed.csv} -- Runs 9 aggregated metrics
    \item \texttt{runs\_10/metrics\_fixed.csv} -- Runs 10 aggregated metrics
    \item \texttt{runs\_*/event\_store.pkl.gz} -- Compressed event data per batch
\end{itemize}

%=============================================================================
\section*{Appendix C: Code References}
%=============================================================================

The analysis was performed using:
\begin{itemize}
    \item \texttt{velo\_workflow.py} -- Main simulation workflow
    \item \texttt{gen\_params\_runs*.py} -- Parameter generation scripts
    \item \texttt{LHCB\_Velo\_Toy\_Models/toy\_validator.py} -- Track validation
    \item \texttt{track\_density\_study\_*.ipynb} -- Analysis notebooks
\end{itemize}

\vspace{1cm}
\hrule
\vspace{0.5cm}
\textit{Report generated: December 2025}\\
\textit{Total events analyzed: $>3000$}\\
\textit{Computing: Nikhef HTCondor cluster}

\end{document}
